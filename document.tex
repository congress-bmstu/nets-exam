\documentclass[12pt]{article}

\usepackage{svg}
\svgpath{{img/}}
\usepackage{wrapfig}

\usepackage[T2A]{fontenc}
\usepackage[utf8]{inputenc}
\usepackage[english,russian]{babel}

\usepackage{amsmath}
\usepackage{amsthm, mathrsfs, mathtools, amssymb}
\usepackage{enumitem}

\usepackage{physics}

\usepackage{epigraph}

\usepackage{tikz}
\usepackage{subcaption}
\usetikzlibrary{decorations.pathmorphing}
% для волнистой линии для фотонов создал стиль линии snake arrow
% рисует волну и на конце ее чуть-чуть прямую линию оставляет для стрелки
\tikzset{snake arrow/.style=
	{->,
		decorate,
		decoration={snake,amplitude=.4mm,segment length=2mm,post length=1mm}},
}
\usepackage{caption}
\usepackage{tensor}
\usepackage{float}
\usepackage{multirow}
\usepackage{multicol}
\usepackage{wrapfig}
\usepackage{hyperref}
\hypersetup{
	colorlinks,
	citecolor=black,
	filecolor=black,
	linkcolor=black,
	urlcolor=black
}

\DeclareMathOperator{\supp}{supp}

\usepackage{geometry}
\geometry{verbose,a4paper,tmargin=1cm,bmargin=2cm,lmargin=1.5cm,rmargin=1.5cm}

\newtheoremstyle{example}% name
{0.7cm}% Space above
{0.7cm}% Space below
{\small}% Body font
{}% Indent amount
{\small\scshape}% Theorem head font
{.}% Punctuation after theorem head
{.5em}% Space after theorem head
{}% Theorem head spec (can be left empty, meaning ‘normal’)

\theoremstyle{example}
\newtheorem{example}{Пример}

\theoremstyle{plain}
\newtheorem{theorem}{Теорема}
\newtheorem{corollary}{Следствие}
\newtheorem*{corollary*}{Следствие} 
\newtheorem{lemma}{Лемма}
\newtheorem{utv}{Утверждение}
\newtheorem*{utv*}{Утверждение}

\theoremstyle{definition}
\newtheorem{definition}{Определение}
\newtheorem*{definition*}{Определение}
\newtheorem{question}{Вопрос}

\theoremstyle{remark}
\newtheorem{remark}{Замечание}
\newtheorem*{remark*}{Замечание}
\numberwithin{remark}{section}

\frenchspacing

\usepackage[labelsep=period]{caption}
\captionsetup{font = small}

\newcommand{\Hom}{\mathrm{Hom}}
\newcommand{\Spl}{\mathrm{Spl}}
\newcommand{\spl}{\mathrm{spl}}

\DeclareMathAlphabet{\mathbfit}{OML}{cmm}{b}{it}


\newcounter{problem} 
\newenvironment{problem}[1][]
{
	\refstepcounter{problem} 
	\par \vspace{0.7em} \noindent
	\textbf{Задача \theproblem}\ifx&#1&\else\ (#1)\fi. 
}
{
	\vspace{1em}	
}

\newenvironment{solution}
{
	\vspace{0.3em}
	\par\textsc{Решение.}
}
{
	\qed
}

\newcommand{\que}[1]{%
	\subsection{#1}
}
\renewcommand{\thesubsection}{\arabic{subsection}}


\begin{document}
	%  \tableofcontents 
	
	\begin{center}
		\LARGE \bf	
		\textsc{Основы сеточных методов}
		\rule{\textwidth}{0.4pt}
	\end{center}
	
	\que{Методы решений линейных разностных уравнений (2-го порядка) с постоянными коэффициентами и специальной правой частью.}

\textit{Линейное разностное уравнение 2-го порядка с постоянными коэффициентами} определяется коэффициентами $a_0, a_1 \in \mathbb{R}$, гдё $a_0 \not = 0$, линейным оператором $\hat{\mathbfit{A}} = \hat{\mathbfit{D}}^2 + a_1 \hat{\mathbfit{D}} + a_0 \hat{\mathbfit{I}} \in \mathrm{End}(V)$, где $\hat{\mathbfit{I}} \in \mathrm{End}(V)$ --- тождественный оператор, для которого $\hat{\mathbfit{I}}(v_{(\cdot)}) = v_{(\cdot)}$, если $v_{(\cdot)} \in V$, и последовательностью $f_{(\cdot)} = (f_n)_{\mathbb{Z}} \in V$. Уравнение определяющиеся такими характеристиками, имеет вид: 
\begin{gather*}
	\hat{\mathbfit{A}}(\mathbfit{y}_{(\cdot)}) = \left(\hat{\mathbfit{D}}^2 + a_1 \hat{\mathbfit{D}} + a_0 \hat{\mathbfit{I}}\right)(\mathbfit{y}_{(\cdot)}) = \hat{\mathbfit{D}}^2(\mathbfit{y}_{(\cdot)}) + a_1 \hat{\mathbfit{D}}(\mathbfit{y}_{(\cdot)}) + a_0 \hat{\mathbfit{I}}(\mathbfit{y}_{(\cdot)}) = \mathbfit{f}_{(\cdot)} \Leftrightarrow \\
	\Leftrightarrow y_{n + 2} + a_1 y_{n + 1} + a_0 y_n = f_n; \quad n \in \mathbb{Z}. 
\end{gather*} 

\dots 

%TODO: Кутыркин отдельно рассматривает решение однородных и потом говорит о виде частного решения, поэтому хз какие <<методы>> тут есть..? Переписать Кутыркина
	\que{Понятия сетки на отрезке, её шага, (равномерные и центрально равномерные сетки), сеточной функции, схемы сеток на отрезке, схемы сеточных функций; сеточного отображения для линейного пространства функций, определённых на отрезке, и схемы сеточных отображений, индуцированной схемой сеток на отрезке.}

Рассмотрим на отрезке $[a, b]$ попарно различные точки $\tau_0, \tau_1, \dots, \tau_k \in [a, b]$.

\begin{definition}
	Список точек $A = \Big<\tau_0, \tau_1, \dots, \tau_k\Big> \subset [a, b]$ называют \textit{сеткой отрезка} $[a, b]$. Указанные в сетке $A$ точки отрезка называют \textit{узлами сетки} $A$. \textit{Шагом сетки} называют число $\mathrm{stp}(A) = \max\{\tau_0 - a, \tau_1 - \tau_0, \tau_2 - \tau_1, \dotsc, \tau_k - \tau_{k - 1}, b - \tau_k\}$. Если выполняются условия $a = \tau_0, \tau_1 - \tau_0 = \tau_2 - \tau_1 = \dotsc = \tau_k - \tau_{k - 1}, \tau_k = b$, то такую сетку называют \textit{равномерной}. В этом случае равномерная сетка $A$ индуцирует \textit{центрально-равномерную сетку} $B = \Big<\frac{\tau_1 + \tau_0}{2}, \frac{\tau_2 + \tau_1}{2}, \dotsc, \frac{\tau_k + \tau_{k - 1}}{2}\Big>$, для которой $\mathrm{stp}(A) = \mathrm{stp}(B) = \frac{b - a}{k}$. 
	
	Последовательность $A_{(\cdot)} = (A_k)_{\mathbb{N}}$ сеток отрезка $[a, b]$ называют \textit{схемой сеток отрезка} $[a, b]$, если $\lim\limits_{k \to +\infty}{\mathrm{stp}(A_k)} = 0$.
	
	Если каждому узлу $\tau_i (i = \overline{1, k})$ сетки $A = \Big<\tau_0, \tau_1, \dotsc, \tau_k\Big> \subset [a, b]$ отрезка $[a, b]$ поставлено в соответствие единственное число $y^i \in \mathbb{R}$, то говорят, что задана $A$-сеточная функция, для которой, благодаря контексту используем обозначение $\tensor[^{>}]{\mathbf{y}}{} = [y^0, y^1, \dotsc, y^k\Big> \in \tensor[^{>}]{\underline{\mathbb{R}}}{^{\abs{A}}}(A)$, где $\abs{A} = k + 1$ и $\tensor[^{>}]{\underline{\mathbb{R}}}{^{\abs{A}}}(A)$ --- нормированное пространство таких $A$-сеточных функций с чебышёвской нормой.  
	
	Если задана схема сеток $A_{(\cdot)} = (A_k)_{\mathbb{N}}$ отрезка $[a, b]$, то последовательность сеточных функций $\tensor[^{>}]{\mathbf{y}}{_{(\cdot)}} = (\tensor[^{>}]{\mathbf{y}}{_{(k)}} \in \tensor[^{>}]{\underline{\mathbb{R}}}{^{\abs{A_k}}}(A_k))_{\mathbb{N}}$ называют схемой $A_{(\cdot)}$-сеточных функций. 
	
	Для сетки $A = \Big<\tau_0, \tau_1, \dotsc, \tau_k\Big> \subset [a, b]$ и линейного пространства $\underline{\mathbf{B}}([a, b], \underline{\mathbb{R}})$ (с чебышёвской нормой) определённых и ограниченных на отрезке $[a, b]$ функций индуцируется $A$-сеточное отображение $\hat{A}: \underline{\mathbf{B}}([a, b], \underline{\mathbb{R}}) \to \tensor[^{>}]{\underline{\mathbb{R}}}{^{\abs{A}}}(A)$, для которого $\hat{A}(f) = \tensor[^{>}]{\mathbf{f}}{} = [f(\tau_0), f(\tau_1), \dotsc, f(\tau_k)\Big> \in \tensor[^{>}]{\underline{\mathbb{R}}}{^{\abs{A}}}(A)$, если $f \in \underline{\mathbf{B}}([a, b], \mathbb{R})$. Аналогично схемы сеточных функций вводим схему сеточных отображений. 
\end{definition} 
	\que{Разности и основное свойство разделённых разностей. Разделённые разности для равномерной сетки.} 
	\que{Теорема об интерполяционном полиноме Ньютона.}

\begin{theorem}[об интерполяционном полиноме Ньютона]
	\begin{align*}
		f(\tau) &= f(\tau_0) + f(\tau_0, \tau_1) (\tau - \tau_0) + f(\tau_0, \tau_1, \tau_2) (\tau - \tau_0) (\tau - \tau_1) + \\ &+ \dotsc + f(\tau_0, \dotsc, \tau_k) (\tau - \tau_0) \cdot \dotsc \cdot (\tau - \tau_{k - 1}) + f(\tau_0, \tau_k) \cdot \Lambda(A, \tau) = \\ &= L_k(A, \tensor[^{>}]{\mathbf{y}}{})(\tau) + f(\tau_0, \tau_k).
	\end{align*}
	
	$f(\tau, A) \cdot \Lambda(A, \tau) = \frac{f^{(k + 1)} (\xi(\tau))}{(k + 1)!} \cdot \Lambda(A, \tau) = R_k(\tau)$ --- значение $k$-го остаточного члена $A$-интерполяции Лагранжа для $\tensor[^{>}]{\mathbf{y}}{}$ в форме Коши, где $\xi(\tau) \in [a, b]$.
\end{theorem}

%TODO: чета последнее предложение не понял совсем. Это у Маши стырено. 
	\que{Разностные производные, их шаблоны и порядок аппроксимирования. Примеры обыкновенных центральных разностных производных 1-го и 2-го порядков.}

Для приближённого вычисления производных от достаточно гладких на отрезке функций используются разностные производные, являющиеся линейными формами от соответствующих сеточных функций. 

Пусть $f$ --- достаточно гладкая функция на отрезке $[a, b]$, включающем точку $\tau_0 \in [a, b]$. Рассмотрим на числовой прямой $\underline{\mathbb{R}}$ равномерную сетку $A_h = \Big<\tau_j = jh : j \in \mathbb{Z}\Big>$ шага $\mathrm{step}(A_h) = h$ и её подсетку $B_h = \Big<\tau_{-q}, \tau_{-q + 1, \dotsc, \tau_0, \dotsc, \tau_p}\Big> \subset [a, b]$, индуцирующую сеточную функцию $\tensor[^{>}]{\mathbf{f}}{} = \Big<f(\tau_{-q}), f(\tau_{-q + 1}), \dotsc, f(\tau_0), \dotsc, f(\tau_p)\Big> \in \tensor[^{>}]{\mathbb{R}}{^{\abs{B_h}}}(B_h)$.

\begin{definition}
	\textit{Разностной производной} $m$-го порядка для $m$-ой производной  $f^{(m)}(\tau_0)$ функции $f$ в точке $\tau_0 \in [a, b]$ называют линейную форму от сеточной функции $\tensor[^{>}]{\mathbf{f}}{} \in \hat{B}_h(f)$ вида:
	\begin{equation*}
		\hat{D}(B_h, \hat{B}_h(f)) = c_{-q}(h) f(\tau_{-q}) + c_{-q + 1}(h)f(\tau_{-q + 1}) + \dotsc + c_0(h) f(\tau_0) + \dotsc + c_p(h) f(\tau_p), 
	\end{equation*}
	где $c_j(h)$ --- числовой коэффициент, зависящий от шага сетки $\mathrm{step}(B_h) = h$ для $j = \overline{-q, p}$. 
	
	Если для разостной формулы выполняется условие:
	\begin{equation*}
		f^{(m)}(\tau_0) = \hat{D}(B_h, \hat{B}_h(f)) + O(h^{\alpha}) \text{ при } h \to 0,
	\end{equation*}
	то число $\alpha > 0$ называют \textit{порядком аппроксимации разностной производной} для производной $f^{(m)}(\tau_0)$ функции $f$ в узле $\tau_0 \in [a, b]$, нахываемым \textit{главным узлом сетки} $B_h$ для разностной производной. 
	
	\begin{example}[разностных формул]
		\begin{align*}
			&\frac{f_i - f_{i - 1}}{h} = \left(\dv{f}{\tau}\right)_{\tau_i} + O(h) \text{ при } h \to 0; \\
			&\frac{f_{i + 1} - f_{i - 1}}{h} = \left(\dv{f}{\tau}\right)_{\tau_i} + O(h^2) \text{ при } h \to 0; \\
			&\frac{f_{i + 1} - 2 f_i - f_{i - 1}}{h} = \left(\dv[2]{f}{\tau}\right)_{\tau_i} + O(h^2) \text{ при } h \to 0,
		\end{align*}
		где $f(\tau_{i - 1}) = f_{i - 1}, f(\tau_i) = f_i, f(\tau_{i + 1}) = f_{i + 1}$. 
	\end{example}
\end{definition} 
	\que{Табуляции и схема табуляций банахова пространства. Сеточная табуляция и сеточное табулирование пространства непрерывных на отрезке функций.}

\begin{definition}
	В линейном арифметическом пространстве $\tensor[^{>}]{\mathbb{R}}{^k}$ введем норму $\norm{\cdot}_{0}$. Тогда нормированное пространство $U = (\tensor[^{>}]{\mathbb{R}}{^k}, \norm{\cdot}_{0})$ называем \textit{табличным пространством}. 
	
	Последовательность табличных пространств $U_{(\cdot)} = \left(U_{(n)} = (\tensor[^{>}]{\mathbb{R}}{^k}), \norm{\cdot}_{(n)}\right)_{\mathbb{N}}$ называют \textit{схемой табличных пространств}, если $\lim\limits_{n \to +\infty}{\mathrm{dim}(U_{(n)})} = +\infty$.
\end{definition}

Рассмотрим банахово пространство $Y_0 = (Y_0, \norm{\cdot})$ и схему табличных пространств $U_{(\cdot)} = \left(U_{(n)} = (\tensor[^{>}]{\mathbb{R}}{^{k_n}}), \norm{\cdot}_{(n)}\right)_{\mathbb{N}}$. Кроме того, предполагаем, что для каждой $k \in \mathbb{N}$ определён эпиморфизм $T_{(k)} \in \Hom_c(Y_0, U_{(k)})$.

\begin{definition}
	Для каждого $k \in \mathbb{N}$ эпиморфизм $T_{(k)} \in \Hom_c(Y_0, U_{(k)})$ называют \textit{табуляцией банахова пространства} $Y_0 = (Y_0, \norm{\cdot})$.
	
	Последовательность эпиморфизмов $T_{(\cdot)} = \left(T_{(k)} \in \Hom_c(Y_0, U_{(k)})\right)_{\mathbb{N}}$ называют \textit{схемой табуляций (табулированием) банахова пространства} $Y_0 = (Y_0, \norm{\cdot})$, если последовательность норм $(\norm{T_{(k)}})_{\mathbb{N}}$ ограничена и для любого $y_0 \in Y_0$ выполняется условие: $\norm{y_0} = \lim\limits_{k \to +\infty}{\norm{T_{(k)}(y_0)}}$.
\end{definition}

\begin{utv}
	Для чебышёвского банахового пространства $Y_0 = \underline{C}([a, b], \underline{\mathbb{R}})$ рассмотрим на отрезке $[a, b]$ схему сеток $A_{(\cdot)} = (A_{k} = \big<\tau_0^k, \tau_1^k, \dotsc, \tau_k^k\big>)_{\mathbb{N}}$. Для каждого $k \in \mathbb{N}$ определим $A_k$-сеточный эпиморфизм $T_{(k)} = \hat{A}_k \in \Hom_c(Y_0, U_{(k)} = \tensor[^{>}]{\mathbb{R}}{^{\abs{A}}}(A))$, то есть для любого $y_0 \in Y_0$ полагаем, что $T_{(k)}(y_0) = \hat{A}_k(y_0) = [y_0(\tau_0^k), y_0(\tau_1^k), \dotsc, y_0(\tau_k^k)\big>$. Тогда вследствии \textit{теоремы Кантора} о равномерной непрерывности функции непрерывной на отрезке, последовательность $T_{(\cdot)} = (T_{(k)})_{\mathbb{N}}$ будет схемой табуляций (табулированием) банахова пространства $Y_0 = \underline{C}([a, b], \underline{\mathbb{R}})$. 
\end{utv} 
	\que{База, базис аппроксимации и схема аппроксимации (аппроксимирования) линейного многообразия в банаховом пространстве; сходимость, аналитическая корректность и корректность аппроксимирования. Примеры корректного и расходящегося аппроксимирования.}

В банаховом пространстве $Y_0 = (Y_0, \norm{\cdot})$ рассмотрим сепарабельное линейное многообразие $X_0 = (X_0, \norm{\cdot}) \subseteq Y_0$. Кроме того, в пространстве $Y_0 = (Y_0, \norm{\cdot})$ введена конечно линейно независимая система (базис) элементов $H = \big<h_0, h_1, \dotsc, h_k\big>$ и задан эпиморфизм $\hat{p}_k \in \Hom_c(Y_0, [H])$, где $[H]$ --- линейная оболочка системы $H = \big<h_0, h_1, \dotsc, h_k\big>$. Тогда эпиморфизм $\hat{p} \in \Hom_c(Y_0, [H])$ называют аппроксимацией линейного многообразия $X_0 \subseteq Y_0$ и значение эпиморфизма $\hat{p}(x_0)$ --- аппроксимацией ($\hat{p}$-аппроксимацией) элемента $x_0 \in X_0$. 

\begin{definition}
	В пространстве $Y_0 = (Y_0, \norm{\cdot})$ рассмотрим последовательность $H_{(\cdot)} = (H_k \subset Y_0)_{\mathbb{N}}$ конечных линейно независимых систем (базисов).
	
	Последовательность $H_{(\cdot)} = (H_k \subset Y_0)_{\mathbb{N}}$ называют \textit{базой аппроксимации} (элементов) линейного многообразия $X_0 \subseteq Y_0$, если для любого элемента $x_0 \in X_0$ существует такая последовательность $y_{(\cdot)} = (y_k \subset [H_k])_{\mathbb{N}}$, для которой $\lim\limits_{k \to +\infty}{y_k} = x_0$ (это свойство возможно обеспечивать только в том случае, когда линейное многообразие $X_0 \subseteq Y_0$ --- сепарабельно). 
	
	Линейно независимую систему $g_{(\cdot)} = (g_{k - 1} \in Y_0)_{\mathbb{N}}$ называют \textit{базисом аппроксимации} пространства $X_0 \subseteq Y_0$, если последовательность $G_{(\cdot)} = (G_k = \big<g_0, g_1, \dotsc, g_k\big>)_{\mathbb{N}}$ является базой аппроксимации пространства $X_0 \subseteq Y_0$.
\end{definition}

\begin{definition}
	Пусть $H_{(\cdot)} = (H_k \subset Y_0)_{\mathbb{N}}$ --- база аппроксимации линейного многообразия $X_0 \subseteq Y_0$ и задана последовательность $\hat{p}_{(\cdot)} = (\hat{p}_k \in \Hom_c(Y_0, [H_k]))_{\mathbb{N}}$ аппроксимациий его элементов. 
	
	Последовательность $\hat{p}_{(\cdot)}$ называют \textit{схемой аппроксимациий} (аппроксимированием) элементов линейного многообразия $X_0 \subset Y_0$, --- \textit{устойчивой}, если последовательность $(\norm{\hat{p}_k})_{\mathbb{N}}$ --- ограничена, --- \textit{сходящейся}, если для любого элемента $x_0 \in X_0$ существует предел $\lim\limits_{k \to +\infty}{\hat{p}_k(x_0)} \in X_0$, --- \textit{аналитически корректной}, если она является сходящейся и $\lim\limits_{k \to +\infty}{\hat{p}_k(x_0)} = x_0$ для любого элемента $x_0 \in X_0$. Аналитически корректную и устойчивую схему аппроксимаций называют корректной схемой аппроксимации. 
\end{definition}

\begin{example}
	\textbf{Расходящееся аппроксимирование. } Рассмотрим в банаховом пространстве $Y_0 = \underline{C}([a, b], \underline{\mathbb{R}})$ линейное многообразие $X_0 = \underline{C}^1([a, b], \underline{\mathbb{R}}) \subset Y_0$ --- гладких на отрезке $[a, b]$ функций. В качестве базы аппроксимации линейного могообразия $X_0 \subset Y_0$ выберем последовательность $H_{(\cdot)} = \left(H_k = \big<h_0 \equiv 1, h_1 = \tau, h_2 = \tau^2, \dotsc, h_k = \tau^k\big> \subset Y_0\right)_{\mathbb{N}}$. Согласно \textit{теореме Веерштраса} (формулировка: любая непрерывная на отрезке функция допускает сколь угодно точное приближение с помощью полинома), эта последовательность является базой аппроксимации пространства $X_0 \subset Y_0$. Схему аппроксимаций многообразия $X_0 \subset Y_0$ зададим с помощью последовательности схемы сеток $A_{(\cdot)} = (A_k)_{\mathbb{N}}$ отрезка $[a, b]$ и последовательности $\hat{p}_{(\cdot)} = (\hat{p}_k \in \Hom_c(Y_0, [H_k]))_{\mathbb{N}}$ следующим образом. Для гладкой функции $x_0 \in X_0$ полагаем, что $\hat{p}_k(x_0) = L(A_k, \hat{A}_k(x_0))$ --- интерполяционный полином Лагранжа для сеточной функции $\hat{A}_k(x_0) \in \tensor[^{>}]{\mathbb{R}}{^{\abs{A_k}}}(A_k)$, если $k \in \mathbb{N}$. Тогда, в общем случае такое аппроксимирование Лагранжа не будет даже сходящимся.
	
	\textbf{Корректное аппроксимирование. } Для банахова пространства $Y_0 = \underline{C}([a, b], \underline{\mathbb{R}})$ рассмотрим схему сеточно-сплайновой аппроксимации $\hat{p}_{(\cdot)} = (\hat{p}_k \in \Hom_c(Y_0, [H_k]))_{\mathbb{N}}$, которая индуцирована схемой $A_{(\cdot)} = (A_k = \big<a = \tau_0^k, \tau_1^k, \dotsc, \tau_k^k = b\big>)_{\mathbb{N}}$ сеток отрезка $[a, b]$ следующим образом. Для $k \in \mathbb{N}$ базис $H_k = (h_0^k, h_1^k, \dotsc, h_k^k) \subset Y_0$ является базисом пространства сплайнов $\Spl_1(A_k)$ первой степени дефекта 1, т.е. для номеров $j = \overline{0, k}$ и функции $y_0 \in Y_0$ следует, что: 
	\begin{equation*}
		\hat{p}_k(y_0) = \spl_1(A_k, \hat{A}_k(y_0)) = \hat{H}_k \circ \hat{A}_k(y_0) = \sum \limits_{j = 0}^{k} y_0(\tau_j^k) h_j^k,
	\end{equation*} 
	где $\hat{H}_k : \tensor[^{>}]{\mathbb{R}}{^{\abs{A_k}}}(A_k) \to [H_k]$ --- $H_k$\textit{-базисный изометричный (интерполяционный) изоморфизм}, для которого $\hat{H}_k(\hat{A}_k(y_0)) = \sum\limits_{j = 0}^{k} y_0(\tau^k_j) h^k_j$. Следовательно, если $k \in \mathbb{N}$, то для сеточной функции $\tensor[^{>}]{\mathbfit{u}}{_{(k)}} \in \tensor[^{>}]{\mathbb{R}}{^{\abs{A_k}}}(A_k)$ выполняются равенства $\hat{A}_k \circ \hat{H}_k (\tensor[^{>}]{\mathbfit{u}}{_{(k)}}) = \tensor[^{>}]{\mathbfit{u}}{_{(k)}}$ и $\norm{\tensor[^{>}]{\mathbfit{u}}{_{(k)}}} = \norm{\hat{H}_k(\tensor[^{>}]{\mathbfit{u}}{_{(k)}})}$, поскольку $H_k$ изометричный (интерполяционный) изоморфизм. В силу \textit{теоремы Кантора} о равномерной непрерывности непрерывной на отрезке функции, такая схема аппроксимации $\hat{p}_{(\cdot)}$ пространства $Y_0 = \underline{C}([a, b], \underline{\mathbb{R}})$ является корректной. 
\end{example} 
	\que{Обоснование сеточного подхода для численных методов.}

Рассмотрим на примере чебышёвского банахова пространства $Y_0 = \underline{C}([a, b], \underline{\mathbb{R}})$ непрерывных на отрезке $[a, b]$ функций возможность корректной замены элементов этого пространства на соответствующие схемы сеточных функций, инюуцированных схемой сеток $$A_{(\cdot)} = \left(A_k = \big<a = \tau_0^k, \tau_1^k, \dotsc, \tau_k^k = b\big>\right)_{\mathbb{N}}$$ отрезка $[a, b]$. Для этого введем на пространстве $Y_0$ схему сеточно-сплайновых аппроксимаций $\hat{p}_(\cdot) = \left(\hat{p}_k = \hat{H}_k \circ \hat{A}_k \in \Hom_c(Y_0, [H_k])\right)$  (из примера корректного аппроксимирования в прошлом билете). 

Будем говорить, что схема $A_{(\cdot)}$-сеточных функций $\tensor[^{>}]{\mathbfit{u}}{_{(\cdot)}} = \left(\tensor[^{>}]{\mathbfit{u}}{_{(k)}} \in \tensor[^{>}]{\mathbb{R}}{^{\abs{A_k}}}(A_k)\right)_{\mathbb{N}}$ ассоциирована с элементом $y_0 \in Y_0$, если для схемы $\tensor[^{>}]{\mathbfit{u}}{_{(\cdot)}}$ схема интерполяций $(\hat{H}_k(\tensor[^{>}]{\mathbfit{u}}{_{(k)}}) = \spl_1(A_k, \tensor[^{>}]{\mathbfit{u}}{_{(k)}}) \in [H_k])_{\mathbb{N}}$ является сходящейся и выполняется условие: $\lim\limits_{k \to +\infty}{\hat{H}_k(\tensor[^{>}]{\mathbfit{u}}{_{(k)}})} = y_0$. Следовательно, для схемы $\tensor[^{>}]{\mathbfit{u}}{_{(\cdot)}}$ справедливо равенство $\lim\limits_{k \to +\infty}{\norm{\tensor[^{>}]{\mathbfit{u}}{_{(k)}}}} = \norm{y_0}$.

Пусть $W_0(A_{(\cdot)})$ --- естественное линейное пространство всех тех и только тех схем $A_{(\cdot)}$-сеточных функций, которые ассоциированы с некоторым элементом пространства $Y_0$. В этом пространстве две схемы $\tensor[^{>}]{\mathbfit{u}}{_{(\cdot)}}, \tensor[^{>}]{\mathbfit{v}}{_{(\cdot)}} \in W_0(A_{(\cdot)})$ ассоциированы с одной функцией $y_0 \in Y_0$ тогда и только тогда, когда схема $\tensor[^{>}]{\mathbfit{u}}{_{(\cdot)}} - \tensor[^{>}]{\mathbfit{v}}{_{(\cdot)}} \in W_0(A_{(\cdot)})$ ассоциирована с нулевой функцией пространства $Y_0$ и в этом случае выполняются равенства $\lim\limits_{k \to +\infty}{\norm{\tensor[^{>}]{\mathbfit{u}}{_{(k)}}}} = \lim\limits_{k \to +\infty}{\norm{\tensor[^{>}]{\mathbfit{v}}{_{(k)}}}} = \norm{y_0}$. 

Рассмотрим эпиморфизм $\Lambda \in \Hom_c(W_0(A_{(\cdot)}), Y_0)$, который схеме $\tensor[^{>}]{\mathbfit{u}}{_{(\cdot)}} \in W_0(A_{(\cdot)})$ ставит в соответствие функцию $\Lambda(\tensor[^{>}]{\mathbfit{u}}{_{(k)}}) = \lim\limits_{k \to +\infty}{\hat{H}_k(\tensor[^{>}]{\mathbfit{u}}{_{(k)}})} \in Y_0$. Очевидно, что ядро $\mathrm{Ker}(\Lambda)$ эпиморфизма состоит из тех и только тех схем пространства $W_0(A_{(\cdot)})$, которые ассоциированы с тождественно нулевой функцией пространства $Y_0 = \underline{C}([a, b], \underline{\mathbb{R}})$. Поскольку $\Lambda$ --- эпиморфизм, то, в силу \textit{теоремы о гомоморфизме} из курса \textit{Линейной алгебры}, фактор-пространство $W_0(A_{(\cdot)}) / \mathrm{Ker}(\Lambda)$ изоморфно пространству $Y_0$. Для класса эквивалентности $\tensor[^{>}]{\widetilde{\mathbfit{u}}}{_{(\cdot)}} \in W_0(A_{(\cdot)}) / \mathrm{Ker}(\Lambda)$ с представителем $\tensor[^{>}]{\mathbfit{u}}{_{(\cdot)}} \in W_0(A_{(\cdot)})$, ассоциированным с функцией $y_0 \in Y_0$, по определению, полагаем, что его норма $\norm{\tensor[^{>}]{\widetilde{\mathbfit{u}}}{_{(\cdot)}}} = \norm{y_0} = \lim\limits_{k \to +\infty}{\norm{\tensor[^{>}]{\mathbfit{u}}{_{(k)}}}}$. Очевидно, что такое определение нормы в пространстве $W_0(A_{(\cdot)})/\mathrm{Ker}(\Lambda)$ --- корректно. Отсюда следует, что пространства $W_0(A_{(\cdot)}) / \mathrm{Ker}(\Lambda)$ и $Y_0$ --- изометрично изоморфны. Аналогичный изометричный изоморфизм можно построить для других нормированных функциональных пространств. Следовательно, при численном решении совместных уравнений (задач) в функциональном нормированном пространстве с помощью схем из конечномерных сеточных аналогов можно ограничиться анализом этих схем сеточных аналогов. В этом случае для схем конечномерных сеточных аналогов уравнений (задач) естественным образом вводятся понятия устойчивости, аналитической корректности и корректности, не используя перед названиями этих терминов прилагательного <<сеточная>>.
 
	\que{Табулирование линейного оператора в пространстве непрерывных на отрезке функций. Пример метода конечных сумм для приближённого вычисления значения интегрального оператора с аналитически заданным гладким ядром.} 
	\que{Аппроксимирование схемой таблично-сеточных аналогов линейного уравнения. Понятие устойчивости схемы таблично-сеточных аналогов уравнения.} 
	\que{Теорема о признаке корректности схемы таблично-сеточных аналогов уравнения, теорема Лакса. Пример метода конечных сумм для численного решения интегрального уравнения Фредгольма 2-го рода.} 
	\que{Метод коллокаций для численного решения линейных уравнений в пространстве гладких функций на отрезке. } 
	\que{Метод прогонки для решения СЛАУ с трёхдиагональной матрицей, имеющей диагональное преобладание.} 
	\que{Конечно разностная схема численного решения краевой задачи (условия 1-го рода) для обыкновенного линейного дифференциального уравнения 2-го порядка.} 
	\que{Метод ломаных Эйлера для численного решения задачи Коши с нормальным ОДУ.} 
	\que{Методы Рунге-Кутта численного решения задачи Коши для нормального ОДУ. Пример методов Рунге-Кутта 2-го порядка.} 
	\que{Методы Рунге-Кутта порядка численного решения задачи Коши для нормальной системы ОДУ и для ОДУ, разрешённого относительно старшей производной.} 
	\que{Многошаговые методы Адамса-Башфорта для численного решения задачи Коши с нормальным ОДУ, пример двухшагового метода Адамса-Башфорта.} 
	\que{Многошаговые методы Адамса-Моултона для численного решения задачи Коши с нормальным ОДУ.} 
	\que{Метод Адамса-Моултона-Башфорта прозгноза и коррекции.} 
	\input{teor-21} 
	\input{teor-22} 
	\input{teor-23} 
	\input{teor-24} 
	\que{Экспоненциальный базис в пространстве $\mathrm{L}_2\left([0, 2 \pi], \mathbb{C}\right)$.} 
	\que{Двуслойные линейные операторы слоистой структуры, спектральный признак устойчивости.} 
	\que{Спектральный признак устойчивости в явной рахностной схеме для одномерного параболического уравнения.} 
	\input{teor-28} 
	\que{Спектральный признак устойчивости при использовании полунеявной схемы для численного решения двумерного параболического уравнения.} 
	\que{Спектральный признак устойчивости при использовании неявной схемы для численного решения двумерного параболического уравнения.} 
	\que{Метод дробных шагов (две одномерные прогонки) для численного решения двумерного параболического уравнения.} 
	\que{Метод дробных шагов (матричная и одномерная прогонка) для численного решения трёхмерного параболического уравнения.} 
	\input{teor-33} 
	\que{Спектральный признак устойчивости в явной разностной схеме для трёхмерного параболического уравнения.} 
	\input{teor-35} 
	\input{teor-36} 
	\que{Спектральный признак устойчивости с неявной разностной схемой (шаблон --- в виде буквы <<Т>>) для одномерного гиперболического уравнения.} 
	\que{Спектральный признак устойчивости в правой явной разностной схеме для простейшего одномерного уравнения переноса.} 
	\que{Спектральный признак устойчивости в левой явной разностной схеме для простейшего одномерного уравнения переноса.} 
	\que{Спектральный признак устойчивости разностной схемы Лакса для численного решения простейшего одномерного уравнения переноса.} 
	\que{Спектральный признак устойчивости разностной схемы Лакса-Вендроффа для численного решения простейшего одномерного уравнения переноса.} 
	\que{Спектральный признак устойчивости в разностной схеме Лакса-Вендроффа при двумерном векторном представлении для простейшего одномерного гиперболического уравнения.} 
	\input{teor-43} 
	\input{teor-44} 
	\que{Пример использования принципа замороженных коэффициентов для одномерного параболического уравнения с переменным коэффициентом диффузии.} 
	\que{Итерационный сеточный метод для численного решения стационарного двумерного уравнения теплопроводности (задача Дирихле с граничными условиями на квадрате $[0; 2] \times [0; 2]$).} 
	\que{Доказательство устойчивости разностной схемы для задачи Дирихле с граничными условиями на квадрате с помощью принципа максимума.} 
	\que{Сеточный метод установления для двумерной задачи Дирихле.} 
	\que{Существование и единственность решения СЛАУ с трёхдиагональной матрицей, удовлетворяющей условию квази-диагонального преобладания.} 
	\que{Принцип максимума для СЛАУ со специальной трёхдиагональной матрицей, удовлетворяющей условию квази-диагонального преобладания, и его следствие для численного сеточного решения стационарного одномерного уравнения теплопроводности с постоянным коэффициентом теплопроводности.} 
	\que{Теорема сравнения для конечно разностного оператора краевой задачи одномерного стационарного уравнения теплопроводности с постоянным коэффициентом теплопроводности.} 
	\que{Конечно-разностная схема как аналога краевой задачи для одномерного стационарного уравнения теплопроводности с постоянным коэффициентом теплопроводности неотрицательной теплоотдачей, корректность этой схемы.} 
	\que{Вид конечно-разностной схемы при численном решении краевой задачи для одномерного (двумерного) стационарного уравнения теплопроводности с переменным коэффициентом теплопроводности.} 
	\input{teor-54} 
	\input{teor-55} 
	\input{teor-56} 
	
\end{document}
