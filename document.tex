\documentclass[12pt]{article}

\usepackage{svg}
\svgpath{{img/}}
\usepackage{wrapfig}

\usepackage[T2A]{fontenc}
\usepackage[utf8]{inputenc}
\usepackage[english,russian]{babel}

\usepackage{amsmath}
\usepackage{amsthm, mathrsfs, mathtools, amssymb}
\usepackage{enumitem}

\usepackage{physics}

\usepackage{epigraph}

\usepackage{tikz}
\usepackage{subcaption}
\usetikzlibrary{decorations.pathmorphing}
% для волнистой линии для фотонов создал стиль линии snake arrow
% рисует волну и на конце ее чуть-чуть прямую линию оставляет для стрелки
\tikzset{snake arrow/.style=
	{->,
		decorate,
		decoration={snake,amplitude=.4mm,segment length=2mm,post length=1mm}},
}
\usepackage{caption}
\usepackage{tensor}
\usepackage{float}
\usepackage{multirow}
\usepackage{multicol}
\usepackage{wrapfig}
\usepackage{hyperref}
\hypersetup{
	colorlinks,
	citecolor=black,
	filecolor=black,
	linkcolor=black,
	urlcolor=black
}

\DeclareMathOperator{\supp}{supp}

\usepackage{geometry}
\geometry{verbose,a4paper,tmargin=1cm,bmargin=2cm,lmargin=1.5cm,rmargin=1.5cm}

\newtheoremstyle{example}% name
{0.7cm}% Space above
{0.7cm}% Space below
{\small}% Body font
{}% Indent amount
{\small\scshape}% Theorem head font
{.}% Punctuation after theorem head
{.5em}% Space after theorem head
{}% Theorem head spec (can be left empty, meaning ‘normal’)

\theoremstyle{example}
\newtheorem{example}{Пример}

\theoremstyle{plain}
\newtheorem{theorem}{Теорема}
\newtheorem{corollary}{Следствие}
\newtheorem*{corollary*}{Следствие} 
\newtheorem{lemma}{Лемма}
\newtheorem{utv}{Утверждение}
\newtheorem*{utv*}{Утверждение}

\theoremstyle{definition}
\newtheorem{definition}{Определение}
\newtheorem*{definition*}{Определение}
\newtheorem{question}{Вопрос}

\theoremstyle{remark}
\newtheorem{remark}{Замечание}
\newtheorem*{remark*}{Замечание}
\numberwithin{remark}{section}

\frenchspacing

\usepackage[labelsep=period]{caption}
\captionsetup{font = small}

\newcommand{\Hom}{\mathrm{Hom}}
\newcommand{\Spl}{\mathrm{Spl}}
\newcommand{\spl}{\mathrm{spl}}

\DeclareMathAlphabet{\mathbfit}{OML}{cmm}{b}{it}


\newcounter{problem} 
\newenvironment{problem}[1][]
{
	\refstepcounter{problem} 
	\par \vspace{0.7em} \noindent
	\textbf{Задача \theproblem}\ifx&#1&\else\ (#1)\fi. 
}
{
	\vspace{1em}	
}

\newenvironment{solution}
{
	\vspace{0.3em}
	\par\textsc{Решение.}
}
{
	\qed
}

\newcommand{\que}[1]{%
	\subsection{#1}
}
\renewcommand{\thesubsection}{\arabic{subsection}}


\begin{document}
	%  \tableofcontents 
	
	\begin{center}
		\LARGE \bf	
		\textsc{Основы сеточных методов}
		\rule{\textwidth}{0.4pt}
	\end{center}
	
	\que{Методы решений линейных разностных уравнений (2-го порядка) с постоянными коэффициентами и специальной правой частью.}
	\que{Понятия сетки на отрезке, её шага, (равномерные и центрально равномерные сетки), сеточной функции, схемы сеток на отрезке, схемы сеточных функций; сеточного отображения для линейного пространства функций, определённых на отрезке, и схемы сеточных отображений, индуцированной схемой сеток на отрезке.}

Рассмотрим на отрезке $[a, b]$ попарно различные точки $\tau_0, \tau_1, \dots, \tau_k \in [a, b]$.

\begin{definition}
	Список точек $A = \Big<\tau_0, \tau_1, \dots, \tau_k\Big> \subset [a, b]$ называют \textit{сеткой отрезка} $[a, b]$. Указанные в сетке $A$ точки отрезка называют \textit{узлами сетки} $A$. \textit{Шагом сетки} называют число $\mathrm{stp}(A) = \max\{\tau_0 - a, \tau_1 - \tau_0, \tau_2 - \tau_1, \dotsc, \tau_k - \tau_{k - 1}, b - \tau_k\}$. Если выполняются условия $a = \tau_0, \tau_1 - \tau_0 = \tau_2 - \tau_1 = \dotsc = \tau_k - \tau_{k - 1}, \tau_k = b$, то такую сетку называют \textit{равномерной}. В этом случае равномерная сетка $A$ индуцирует \textit{центрально-равномерную сетку} $B = \Big<\frac{\tau_1 + \tau_0}{2}, \frac{\tau_2 + \tau_1}{2}, \dotsc, \frac{\tau_k + \tau_{k - 1}}{2}\Big>$, для которой $\mathrm{stp}(A) = \mathrm{stp}(B) = \frac{b - a}{k}$. 
	
	Последовательность $A_{(\cdot)} = (A_k)_{\mathbb{N}}$ сеток отрезка $[a, b]$ называют \textit{схемой сеток отрезка} $[a, b]$, если $\lim\limits_{k \to +\infty}{\mathrm{stp}(A_k)} = 0$.
	
	Если каждому узлу $\tau_i (i = \overline{1, k})$ сетки $A = \Big<\tau_0, \tau_1, \dotsc, \tau_k\Big> \subset [a, b]$ отрезка $[a, b]$ поставлено в соответствие единственное число $y^i \in \mathbb{R}$, то говорят, что задана $A$-сеточная функция, для которой, благодаря контексту используем обозначение $\tensor[^{>}]{\mathbf{y}}{} = [y^0, y^1, \dotsc, y^k\Big> \in \tensor[^{>}]{\underline{\mathbb{R}}}{^{\abs{A}}}(A)$, где $\abs{A} = k + 1$ и $\tensor[^{>}]{\underline{\mathbb{R}}}{^{\abs{A}}}(A)$ --- нормированное пространство таких $A$-сеточных функций с чебышёвской нормой.  
	
	Если задана схема сеток $A_{(\cdot)} = (A_k)_{\mathbb{N}}$ отрезка $[a, b]$, то последовательность сеточных функций $\tensor[^{>}]{\mathbf{y}}{_{(\cdot)}} = (\tensor[^{>}]{\mathbf{y}}{_{(k)}} \in \tensor[^{>}]{\underline{\mathbb{R}}}{^{\abs{A_k}}}(A_k))_{\mathbb{N}}$ называют схемой $A_{(\cdot)}$-сеточных функций. 
	
	Для сетки $A = \Big<\tau_0, \tau_1, \dotsc, \tau_k\Big> \subset [a, b]$ и линейного пространства $\underline{\mathbf{B}}([a, b], \underline{\mathbb{R}})$ (с чебышёвской нормой) определённых и ограниченных на отрезке $[a, b]$ функций индуцируется $A$-сеточное отображение $\hat{A}: \underline{\mathbf{B}}([a, b], \underline{\mathbb{R}}) \to \tensor[^{>}]{\underline{\mathbb{R}}}{^{\abs{A}}}(A)$, для которого $\hat{A}(f) = \tensor[^{>}]{\mathbf{f}}{} = [f(\tau_0), f(\tau_1), \dotsc, f(\tau_k)\Big> \in \tensor[^{>}]{\underline{\mathbb{R}}}{^{\abs{A}}}(A)$, если $f \in \underline{\mathbf{B}}([a, b], \mathbb{R})$. Аналогично схемы сеточных функций вводим схему сеточных отображений. 
\end{definition} 
	\que{Разности и основное свойство разделённых разностей. Разделённые разности для равномерной сетки.}

\begin{definition}
	Список точек $B = (t_0, t_1, \dotsc, t_m)$ отрезка $[a, b]$ называется ($m \in \mathrm{Z}_{+}$) псевдо-сеткой отрезка $[a, b]$ и квази-сеткой $[a, b]$, если $\abs{B} = m + 1$. 
	
	Если $\tau \in [a, b]$ и $B = (t_0, \dotsc, t_m)$ --- квази-сетка, то $(\tau, B) = (\tau, t_0, \dotsc, t_m)$ и $(B, \tau) = (t_0, \dotsc, t_m, \tau)$. 
\end{definition}

\begin{definition}
	Пусть $f \in \underline{C}^{k+1}([a, b], \mathbb{R})$. Тогда для квази-сетки $A = (\tau_0, \dotsc, \tau_k)$. 
	
	\begin{enumerate}[label={\arabic*)}, ref={\arabic*}, start=0]
		\item $f(\tau_0)$ --- $(\tau_0)$-разделённая разность для $f$ 1-го порядка;
		
		\item $f(\tau_0, \tau_1) = \frac{f(\tau_1) - f(\tau_0)}{\tau_1 - \tau_0}$ --- $(\tau_0, \tau_1)$-разделенная разность для $f$ 1-го порядка;
	\end{enumerate}	
	\begin{enumerate}[label={\alph*)}, start=10]
		\item $f(\tau_0, \tau_1, \dotsc, \tau_j) = \frac{f(\tau_1, \dotsc, \tau_j) - f(\tau_0, \tau_1, \dotsc, \tau_{j - 1})}{\tau_j - \tau_0}$
	\end{enumerate}
	
	Если $\tau \in [a, b]$, то:
	\begin{equation*}
		f(A, \tau) = \begin{cases}
					 	\frac{f(\tau_1, \dotsc, \tau_k, \tau) - f(A)}{\tau - \tau_0}, \quad &\tau \not \in A, \\
					 	\lim\limits_{\theta \to \tau}{\frac{f(\tau_1, \dotsc, \tau_k, \theta) - f(A)}{\theta - \tau_0}}, \quad &\tau \in A. 
					 \end{cases}
	\end{equation*}
\end{definition}

%\begin{remark}
%	$\\$
%	\begin{enumerate}[label={\arabic*)}]
%		\item $\Delta y_0 = y_1 - y_0, \Delta y_1 = y_2 - y_1, \dotsc, \Delta y_{k - 1} = y_k - y_{k - 1}$ (разность 1-го порядка);
%	\end{enumerate}
%	\begin{enumerate}[label={\alph*)}, start=11]
%		\item $\Delta^k y_0 = \sum \limits_{i = 0}^{k} (-1)^{i} C^i_k y_{k - i}$ (разность $k$-го порядка)
%	\end{enumerate}
%\end{remark}

\begin{lemma}[о разделенных разностей с равномерной сеткой]
	Для равномерной сетки $A$ на $[a, b]$ и $h = \mathrm{stp}(A)$:
	\begin{equation*}
		\frac{1}{k! \cdot h^{k}} \Delta^{k} y_0 = f(\tau_0, \tau_1, \dotsc, \tau_{k - 1}, \tau_k).
	\end{equation*}
\end{lemma}

\begin{lemma}[о независимости от порядка записи аргументов]
	
	Разделённая разность $f(\tau_0, \dotsc, \tau_k)$ является симметричной функцией своих аргументов.
	
	\begin{equation*}
		\boxed{f(\tau_0, \dotsc, \tau_k) = \sum \limits_{t = 0}^{k} \frac{f(\tau_i)}{(\tau_i - \tau_0) \cdot \dotsc \cdot (\tau_i - \tau_{i - 1}) \cdot (\tau_i - \tau_{i + 1}) \cdot \dotsc \cdot (\tau_i - \tau_k)}}.
	\end{equation*}
\end{lemma}

%TODO: Хз что имеется в виду под <<основным свойством>>. Я так понял лемма2... 
	\que{Теорема об интерполяционном полиноме Ньютона.}

\begin{theorem}[об интерполяционном полиноме Ньютона]
	\begin{align*}
		f(\tau) &= f(\tau_0) + f(\tau_0, \tau_1) (\tau - \tau_0) + f(\tau_0, \tau_1, \tau_2) (\tau - \tau_0) (\tau - \tau_1) + \\ &+ \dotsc + f(\tau_0, \dotsc, \tau_k) (\tau - \tau_0) \cdot \dotsc \cdot (\tau - \tau_{k - 1}) + f(\tau_0, \tau_k) \cdot \Lambda(A, \tau) = \\ &= L_k(A, \tensor[^{>}]{\mathbf{y}}{})(\tau) + f(\tau_0, \tau_k).
	\end{align*}
	
	$f(\tau, A) \cdot \Lambda(A, \tau) = \frac{f^{(k + 1)} (\xi(\tau))}{(k + 1)!} \cdot \Lambda(A, \tau) = R_k(\tau)$ --- значение $k$-го остаточного члена $A$-интерполяции Лагранжа для $\tensor[^{>}]{\mathbf{y}}{}$ в форме Коши, где $\xi(\tau) \in [a, b]$.
\end{theorem}

%TODO: чета последнее предложение не понял совсем. Это у Маши стырено. 
	\que{Разностные производные, их шаблоны и порядок аппроксимирования. Примеры обыкновенных центральных разностных производных 1-го и 2-го порядков.} 
	\que{Табуляции и схема табуляций банахова пространства. Сеточная табуляция и сеточное табулирование пространства непрерывных на отрезке функций.}

\begin{definition}
	В линейном арифметическом пространстве $\tensor[^{>}]{\mathbb{R}}{^k}$ введем норму $\norm{\cdot}_{0}$. Тогда нормированное пространство $U = (\tensor[^{>}]{\mathbb{R}}{^k}, \norm{\cdot}_{0})$ называем \textit{табличным пространством}. 
	
	Последовательность табличных пространств $U_{(\cdot)} = \left(U_{(n)} = (\tensor[^{>}]{\mathbb{R}}{^k}), \norm{\cdot}_{(n)}\right)_{\mathbb{N}}$ называют \textit{схемой табличных пространств}, если $\lim\limits_{n \to +\infty}{\mathrm{dim}(U_{(n)})} = +\infty$.
\end{definition}

Рассмотрим банахово пространство $Y_0 = (Y_0, \norm{\cdot})$ и схему табличных пространств $U_{(\cdot)} = \left(U_{(n)} = (\tensor[^{>}]{\mathbb{R}}{^{k_n}}), \norm{\cdot}_{(n)}\right)_{\mathbb{N}}$. Кроме того, предполагаем, что для каждой $k \in \mathbb{N}$ определён эпиморфизм $T_{(k)} \in \Hom_c(Y_0, U_{(k)})$.

\begin{definition}
	Для каждого $k \in \mathbb{N}$ эпиморфизм $T_{(k)} \in \Hom_c(Y_0, U_{(k)})$ называют \textit{табуляцией банахова пространства} $Y_0 = (Y_0, \norm{\cdot})$.
	
	Последовательность эпиморфизмов $T_{(\cdot)} = \left(T_{(k)} \in \Hom_c(Y_0, U_{(k)})\right)_{\mathbb{N}}$ называют \textit{схемой табуляций (табулированием) банахова пространства} $Y_0 = (Y_0, \norm{\cdot})$, если последовательность норм $(\norm{T_{(k)}})_{\mathbb{N}}$ ограничена и для любого $y_0 \in Y_0$ выполняется условие: $\norm{y_0} = \lim\limits_{k \to +\infty}{\norm{T_{(k)}(y_0)}}$.
\end{definition}

\begin{utv}
	Для чебышёвского банахового пространства $Y_0 = \underline{C}([a, b], \underline{\mathbb{R}})$ рассмотрим на отрезке $[a, b]$ схему сеток $A_{(\cdot)} = (A_{k} = \big<\tau_0^k, \tau_1^k, \dotsc, \tau_k^k\big>)_{\mathbb{N}}$. Для каждого $k \in \mathbb{N}$ определим $A_k$-сеточный эпиморфизм $T_{(k)} = \hat{A}_k \in \Hom_c(Y_0, U_{(k)} = \tensor[^{>}]{\mathbb{R}}{^{\abs{A}}}(A))$, то есть для любого $y_0 \in Y_0$ полагаем, что $T_{(k)}(y_0) = \hat{A}_k(y_0) = [y_0(\tau_0^k), y_0(\tau_1^k), \dotsc, y_0(\tau_k^k)\big>$. Тогда вследствии \textit{теоремы Кантора} о равномерной непрерывности функции непрерывной на отрезке, последовательность $T_{(\cdot)} = (T_{(k)})_{\mathbb{N}}$ будет схемой табуляций (табулированием) банахова пространства $Y_0 = \underline{C}([a, b], \underline{\mathbb{R}})$. 
\end{utv} 
	\que{База, базис аппроксимации и схема аппроксимации (аппроксимирования) линейного многообразия в банаховом пространстве; сходимость, аналитическая корректность и корректность аппроксимирования. Примеры корректного и расходящегося аппроксимирования.} 
	\que{Обоснование сеточного подхода для численных методов.} 
	\que{Табулирование линейного оператора в пространстве непрерывных на отрезке функций. Пример метода конечных сумм для приближённого вычисления значения интегрального оператора с аналитически заданным гладким ядром.} 
	\que{Аппроксимирование схемой таблично-сеточных аналогов линейного уравнения. Понятие устойчивости схемы таблично-сеточных аналогов уравнения.} 
	\que{Теорема о признаке корректности схемы таблично-сеточных аналогов уравнения, теорема Лакса. Пример метода конечных сумм для численного решения интегрального уравнения Фредгольма 2-го рода.} 
	\que{Метод коллокаций для численного решения линейных уравнений в пространстве гладких функций на отрезке. } 
	\que{Метод прогонки для решения СЛАУ с трёхдиагональной матрицей, имеющей диагональное преобладание.} 
	\que{Конечно разностная схема численного решения краевой задачи (условия 1-го рода) для обыкновенного линейного дифференциального уравнения 2-го порядка.} 
	\que{Метод ломаных Эйлера для численного решения задачи Коши с нормальным ОДУ.} 
	\que{Методы Рунге-Кутта численного решения задачи Коши для нормального ОДУ. Пример методов Рунге-Кутта 2-го порядка.} 
	\que{Методы Рунге-Кутта порядка численного решения задачи Коши для нормальной системы ОДУ и для ОДУ, разрешённого относительно старшей производной.} 
	\que{Многошаговые методы Адамса-Башфорта для численного решения задачи Коши с нормальным ОДУ, пример двухшагового метода Адамса-Башфорта.} 
	\que{Многошаговые методы Адамса-Моултона для численного решения задачи Коши с нормальным ОДУ.} 
	\que{Метод Адамса-Моултона-Башфорта прозгноза и коррекции.} 
	\que{Доказательство условной корректности явной разностной схемы численного решения задачи Коши для одномерного параболического уравнения.} 
	\que{Доказательство безусловной корректности неявной разностной схемы численного решения задачи Коши для одномерного параболического уравнения.} 
	\que{Метод матричной прогонки.} 
	\que{Комплексное гильбертово пространство.} 
	\que{Экспоненциальный базис в пространстве $\mathrm{L}_2\left([0, 2 \pi], \mathbb{C}\right)$.} 
	\que{Двуслойные линейные операторы слоистой структуры, спектральный признак устойчивости.} 
	\que{Спектральный признак устойчивости в явной рахностной схеме для одномерного параболического уравнения.} 
	\que{Спектральный признак устойчивости в явной разностной схеме для двумерного параболического уравнения.} 
	\que{Спектральный признак устойчивости при использовании полунеявной схемы для численного решения двумерного параболического уравнения.} 
	\que{Спектральный признак устойчивости при использовании неявной схемы для численного решения двумерного параболического уравнения.} 
	\que{Метод дробных шагов (две одномерные прогонки) для численного решения двумерного параболического уравнения.} 
	\que{Метод дробных шагов (матричная и одномерная прогонка) для численного решения трёхмерного параболического уравнения.} 
	\que{Спектральный признак устойчивости схемы Кранка-Николсона для одномерного параболического уравнения.} 
	\que{Спектральный признак устойчивости в явной разностной схеме для трёхмерного параболического уравнения.} 
	\que{Спектральный признак устойчивости в неявной разностной схеме для одномерного параболического уравнения.} 
	\que{Спектральный признак устойчивости в явной разностной схеме (шаблон --- <<крест>>) для одномерного гиперболического уравнения.} 
	\que{Спектральный признак устойчивости с неявной разностной схемой (шаблон --- в виде буквы <<Т>>) для одномерного гиперболического уравнения.} 
	\que{Спектральный признак устойчивости в правой явной разностной схеме для простейшего одномерного уравнения переноса.} 
	\que{Спектральный признак устойчивости в левой явной разностной схеме для простейшего одномерного уравнения переноса.} 
	\que{Спектральный признак устойчивости разностной схемы Лакса для численного решения простейшего одномерного уравнения переноса.} 
	\que{Спектральный признак устойчивости разностной схемы Лакса-Вендроффа для численного решения простейшего одномерного уравнения переноса.} 
	\que{Спектральный признак устойчивости в разностной схеме Лакса-Вендроффа при двумерном векторном представлении для простейшего одномерного гиперболического уравнения.} 
	\que{Условия монотонности и устойчивости явной разностной схемы для численного решения одномерного уравнения конвекции-диффузии.} 
	\que{Понятие монотонности (двуслойной) разностной схемы. Теорема Годунова и её следствие о признаке устойчивости монотонной схемы.} 
	\que{Пример использования принципа замороженных коэффициентов для одномерного параболического уравнения с переменным коэффициентом диффузии.} 
	\que{Итерационный сеточный метод для численного решения стационарного двумерного уравнения теплопроводности (задача Дирихле с граничными условиями на квадрате $[0; 2] \times [0; 2]$).} 
	\que{Доказательство устойчивости разностной схемы для задачи Дирихле с граничными условиями на квадрате с помощью принципа максимума.} 
	\que{Сеточный метод установления для двумерной задачи Дирихле.} 
	\que{Существование и единственность решения СЛАУ с трёхдиагональной матрицей, удовлетворяющей условию квази-диагонального преобладания.} 
	\que{Принцип максимума для СЛАУ со специальной трёхдиагональной матрицей, удовлетворяющей условию квази-диагонального преобладания, и его следствие для численного сеточного решения стационарного одномерного уравнения теплопроводности с постоянным коэффициентом теплопроводности.} 
	\que{Теорема сравнения для конечно разностного оператора краевой задачи одномерного стационарного уравнения теплопроводности с постоянным коэффициентом теплопроводности.} 
	\que{Конечно-разностная схема как аналога краевой задачи для одномерного стационарного уравнения теплопроводности с постоянным коэффициентом теплопроводности неотрицательной теплоотдачей, корректность этой схемы.} 
	\que{Вид конечно-разностной схемы при численном решении краевой задачи для одномерного (двумерного) стационарного уравнения теплопроводности с переменным коэффициентом теплопроводности.} 
	\que{Аппроксимация со вторым порядком краевых условий 2-го рода разностной схемы краевой задачи для одномерного стационарного уравнения теплопроводности.} 
	\que{Метод Ритца численного решения одномерной стационарной задачи для уравнения теплопроводности.} 
	\que{Метод Галёркина численного решения краевой задачи для одномерного стационарного уравнения теплопроводности.} 
	
\end{document}
