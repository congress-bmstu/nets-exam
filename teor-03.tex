\que{Разности и основное свойство разделённых разностей. Разделённые разности для равномерной сетки.}

\begin{definition}
	Список точек $B = (t_0, t_1, \dotsc, t_m)$ отрезка $[a, b]$ называется ($m \in \mathrm{Z}_{+}$) псевдо-сеткой отрезка $[a, b]$ и квази-сеткой $[a, b]$, если $\abs{B} = m + 1$. 
	
	Если $\tau \in [a, b]$ и $B = (t_0, \dotsc, t_m)$ --- квази-сетка, то $(\tau, B) = (\tau, t_0, \dotsc, t_m)$ и $(B, \tau) = (t_0, \dotsc, t_m, \tau)$. 
\end{definition}

\begin{definition}
	Пусть $f \in \underline{C}^{k+1}([a, b], \mathbb{R})$. Тогда для квази-сетки $A = (\tau_0, \dotsc, \tau_k)$. 
	
	\begin{enumerate}[label={\arabic*)}, ref={\arabic*}, start=0]
		\item $f(\tau_0)$ --- $(\tau_0)$-разделённая разность для $f$ 1-го порядка;
		
		\item $f(\tau_0, \tau_1) = \frac{f(\tau_1) - f(\tau_0)}{\tau_1 - \tau_0}$ --- $(\tau_0, \tau_1)$-разделенная разность для $f$ 1-го порядка;
	\end{enumerate}	
	\begin{enumerate}[label={\alph*)}, start=10]
		\item $f(\tau_0, \tau_1, \dotsc, \tau_j) = \frac{f(\tau_1, \dotsc, \tau_j) - f(\tau_0, \tau_1, \dotsc, \tau_{j - 1})}{\tau_j - \tau_0}$
	\end{enumerate}
	
	Если $\tau \in [a, b]$, то:
	\begin{equation*}
		f(A, \tau) = \begin{cases}
					 	\frac{f(\tau_1, \dotsc, \tau_k, \tau) - f(A)}{\tau - \tau_0}, \quad &\tau \not \in A, \\
					 	\lim\limits_{\theta \to \tau}{\frac{f(\tau_1, \dotsc, \tau_k, \theta) - f(A)}{\theta - \tau_0}}, \quad &\tau \in A. 
					 \end{cases}
	\end{equation*}
\end{definition}

%\begin{remark}
%	$\\$
%	\begin{enumerate}[label={\arabic*)}]
%		\item $\Delta y_0 = y_1 - y_0, \Delta y_1 = y_2 - y_1, \dotsc, \Delta y_{k - 1} = y_k - y_{k - 1}$ (разность 1-го порядка);
%	\end{enumerate}
%	\begin{enumerate}[label={\alph*)}, start=11]
%		\item $\Delta^k y_0 = \sum \limits_{i = 0}^{k} (-1)^{i} C^i_k y_{k - i}$ (разность $k$-го порядка)
%	\end{enumerate}
%\end{remark}

\begin{lemma}[о разделенных разностей с равномерной сеткой]
	Для равномерной сетки $A$ на $[a, b]$ и $h = \mathrm{stp}(A)$:
	\begin{equation*}
		\frac{1}{k! \cdot h^{k}} \Delta^{k} y_0 = f(\tau_0, \tau_1, \dotsc, \tau_{k - 1}, \tau_k).
	\end{equation*}
\end{lemma}

\begin{lemma}[о независимости от порядка записи аргументов]
	
	Разделённая разность $f(\tau_0, \dotsc, \tau_k)$ является симметричной функцией своих аргументов.
	
	\begin{equation*}
		\boxed{f(\tau_0, \dotsc, \tau_k) = \sum \limits_{t = 0}^{k} \frac{f(\tau_i)}{(\tau_i - \tau_0) \cdot \dotsc \cdot (\tau_i - \tau_{i - 1}) \cdot (\tau_i - \tau_{i + 1}) \cdot \dotsc \cdot (\tau_i - \tau_k)}}.
	\end{equation*}
\end{lemma}

%TODO: Хз что имеется в виду под <<основным свойством>>. Я так понял лемма2...